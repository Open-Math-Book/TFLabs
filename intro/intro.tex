 
\chapter{Introduction}

This book aims to introduce the reader to several computational tools that are useful for people entering careers in Mathematics and the Sciences.  We have chosen to emphasize systems that are open-source but we generally indicate other options in the introduction to each chapter.

The book is divided into 8 main chapters.

\begin{itemize}

\item This Introduction.

\item Writing with embedded mathematical content.

\item Computational Algebra.

\item The creation and inclusion of Graphics in technical documents.

\item Interactive Geometry.

\item A chapter on the use of Spreadsheets.

\item Statistics

\item Miscellaneous

\end{itemize}

In each of those chapters there will be technological as well as mathematical learning goals.  The point of this book is to collect laboratories/activities that will allow one to reach these goals through an active learning strategy: doing stuff, not being told about it! 

An important aspect of developing a facility with tech is that you get better at it the more you do it.  Many people advocate that ``learning how to learn'' is one of the biggest benefits of a college education.  So we hope that part of the takeaway from doing these labs is that it will be easier for you to figure out other tools that you'll encounter in the future.  For example, a great many computer applications allow you to access their functionalities via a menu system.  Many of those menu entries have a keyboard shortcut associated with them.  Once you see the increased productivity you get from using such shortcuts, you start to look for them in other applications.  And, there are fairly consistent conventions about how those shortcuts are selected so knowledge you gain in one app often transfers over directly to others!

In the early days of computing, someone introduced the initialism RTFM which doesn't really stand for ``Read that fine manual.''  That's one piece of advice that you should certainly take away from the TFLabs experience -- learn how to access and read the help facility!  A couple of other suggestions for getting up to speed quickly in a new computing environment:

\begin{itemize}
	\item Often, the help for a particular command will include examples.  Scrolling past that wall of confusing text until you get to the relevant examples isn't a terrible strategy.
	\item See if there are so-called ``tool-tips'' -- hovering your mouse over a button often causes a pop-up that displays a hint about what the button does.
	\item Look for the aforementioned keyboard shortcuts.  A lot of programs use Ctrl-Z as a shortcut for ``undo'' (this can be super useful!  Unless the command is Ctrl-U in your particular application.)
	\item Learn to read error messages.  These messages are generated programattically when something goes wrong - this usually makes them a bit cryptic.  If you're able to crack the code, you'll probably at least get a hint as to what caused the problem.
	\item Use your favorite search engine to see if there is relevant help already out there.  ``Hey Siri, how do I get an ampersand in LaTeX?''  (I really don't know if that will work, but putting that question in my browser's search bar sure does!)
	\item Doing the previous ``googling'' will often lead you to online help fora like Quora, Reddit and Stack Overflow.  Reading threads in a forum can be really helpful.  Posting questions in a forum is hit-or-miss -- sometimes you'll get a quick useful answer, other times the result is not what one would want\textellipsis  
\end{itemize}

Let's get started!

