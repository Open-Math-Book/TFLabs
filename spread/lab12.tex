\begin{enumerate}

\item Write out the expanded form of $(x+1)^5$.

\vspace{1in}

\item Use a spreadsheet to create a table of the binomial coefficients $\binom{n}{k} = \frac{n!}{k!(n-k)!}$.  The rows in the table should be left-justified, for instance, column A will contain all of the $1$'s that form the first entries of the rows.  You should create your table twice.  Once using the builtin function $\tt COMBIN()$ (look for it in the function wizard) -- you'll need to think carefully about what should be ``sticky'' in this case!  The second version of the table should use the rule that the entries in Pascal's triangle are the sum of the ones above and on either side of them -- here, you'll need to think carefully about how ``on either side'' morphs when the table is left-justified.
	

\vspace{1in}

\item Revisit the grade spreadsheet from the previous section and insert a new row at the top.  Put the "weights" for homework, 
quizzes and exams into cells in this row.  (As the teacher you might want to experiment with different 
weighting schemes.) Are the final grades changed by much if we change the weights to 10, 
20 and 70 percent respectively?

\vspace{1 in}

\item Use a spreadsheet to do a finite differences analysis of the following sequence:

	  \[ 1 \quad 3 \quad 7 \quad 13 \quad 21 \ldots \]

\vspace{1in}

\item Find the "back diagonals" for the sequences of squares, cubes, 4th  and 5th powers.
(We care about back diagonals because you can use them to generate the entire sequence! (under the assumption that the bottom-most number is a constant) (sorry about all the parentheses)). 

\vfill

\end{enumerate}
