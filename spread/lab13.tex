Returning to our gradebook example$\ldots$

On the TFLabs website, find and download the file ``grades.ods''

In this spreadsheet, the MIN function has been used to ``drop the lowest quiz.''  Is this how you accomplished this in Lab 11?  The cells that contain weights for the various kinds of assessments are also there, and accessed in the formulas for cells using absolute (a.k.a. sticky) references.

\begin{enumerate}

\item Look up the {\tt CHOOSE()} and the {\tt INT()} functions in the function wizard, and figure out how to use them to assign letter grades with $+$ and $-$ modifiers.

\vspace{1in}

\item There are quite a few ``mathematical'' functions available in the function wizard.  For example, many of the functions that start with `A' are inverse trigonometric functions\dots  Figure out how to use ordinary division and the {\tt FLOOR()} function to find the quotient and reaminder you get when dividing one number by another.

\vspace{1in} 

\item Closely related to questions about the division process, there are two important mathematical functions called {\tt GCD()} and {\tt LCM() }.  Play around with these to discover what they do.  Can you determine a formula for {\tt GCD() $\times$ LCM() }?

\vfill

\newpage

\item Use a spreadsheet to model the balance in a loan account.  Use sticky references to two cells that contain the percent rate for the interest on the loan ($r$) and the monthly payment amount ($M$).  You'll also want to decide on the initial amount of the loan (maybe look up the cost of a new car you'd like to own\textellipsis)   Each month, the balance in the account is adjusted in two ways: interest is 
added on, and the payment is deducted.  

\[ A_{new} \quad = \quad (1+r/12) \cdot A_{prev} - M. \]

Setup a spreadsheet that tracks the account balance over 48 months.  Then adjust the value of $M$ until the balance in the final month is as close to zero as you can make it.

A reasonable interest rate for a car loan is around 8\% ($r = 0.08$ as a decimal).

What was your initial guess for $M$?

\vfill

Record the sequence of $M$ values that you tried in zeroing in on the correct amount.

\vfill

What happens to the balance in the account if you set $M$ equal to one month's worth of interest?

\vfill

\end{enumerate}
