{\bf \Large Solving equations}

(Note: follow along with these initial computations then try the problems.)

We will investigate functions of the form $f(x)=ax^2+bx+c$. To begin
with, we will consider when $a=1$, $b=0$, and $c=0$, i.e. $f(x)=x^2$.


Start by defining $f(x)=x^2$ in Sage.

\begin{verbatim}
f(x)=x^2
\end{verbatim}

To solve the equation $f(x)=1$ (i.e. $x^2=1$) in Sage, we can use the
\textbf{solve} command:

\begin{verbatim}
solve(f(x)==1, x)
\end{verbatim}

This will solve the equation $f(x)=1$, solving for the variable $x$.
Note that when solving, we need to put two equals signs to denote
equality. This is because Sage will interpret a single equals sign
as an assignment (i.e. we would be defining $f(x)$ to be the function
that is always equal to $1$).

The output we obtain is

\begin{verbatim}
[x == -1, x == 1]
\end{verbatim}

These are the two solutions that we expect of $x=-1$ and $x=1$.

\subsection{Practice Exercises}

\begin{enumerate}
	\item Try to solve the equation $f(x)=4$ using Sage.
	\item Solve the equation $f(x)=3$ using Sage.
	\item Solve the equation $f(x)=0$ using Sage.
	\item Do you expect for there to be any solutions for $f(x)=-9$?
		Try it in Sage and see what happens.
	\item For what values of $h$ does $f(x)=h$ have two real solutions?
		Only 1 solution? Two imaginary solutions?
	\item Define a new function $g(x) = x^2 + 10x + 21$.  Let's try two things:
	\begin{enumerate}
		\item Use the \verb+factor()+ member function to see how $g(x)$ factors.
		\item Use the \verb+solve()+ command to see when $g(x)$ is equal to zero.
	\end{enumerate}
	Can you explain the minus signs?

	\item Do the things \verb+factor()+ and \verb+solve()+ in the previous problem for 
	\begin{enumerate}
		\item $g(x) = x^2 + 10x + 22$
		\item $g(x) = x^2 + 10x + 23$
		\item $g(x) = x^2 + 10x + 24$
		\item $g(x) = x^2 + 10x + 25$
		\item $g(x) = x^2 + 10x + 26$
	\end{enumerate}

    Sometimes Sage factors the polynomial into linear factors. \newline
    Sometimes Sage is refusing to factor because the zeros are not nice numbers.\newline
    Once, Sage refuses to factor because things have gotten truly weird.

    Identify which is which?  What do you suppose \verb+I - 5+ means?

\end{enumerate}

{\bf \Large Graphing functions}

(Again, follow along with the first few computations, then try the problems.)

Another useful way to analyze things is by visualizing functions using plots of their graphs. Let's begin by plotting the function $f(x)=x^2$:

\begin{verbatim}
f(x)=x^2
plot(f(x))
\end{verbatim}

The first line defines the function $f(x)=x^2$, then the second line will plot
the function. Note that since we did not specify the range for the axes, the
plot will automatically pick a range. If we want to see more of the graph, we
can try adjusting the \textbf{xmin}, \textbf{xmax}, \textbf{ymin}, and \textbf{ymax}
values, which correspond to the minimum and maximum values of the
$x$- and $y$- axes that we desire. For example, changing the plot
command to

\begin{verbatim}
plot(f(x),xmin=-3,xmax=3,ymin=-2,ymax=9)
\end{verbatim}

will gives us an $x$-axis that ranges from $-3$ to $3$ and a $y$-axis that
ranges from $-2$ to $9$.

We can also plot multiple functions on the same plot, in case we
want to compare them together. Let's try overlaying the graph of
the function $g(x)=1$ onto the same set of axes:

\begin{verbatim}
f(x)=x^2
g(x)=1
plot([f(x), g(x)],xmin=-3,xmax=3,ymin=-2,ymax=9)
\end{verbatim}

We have defined two functions now, $f(x)=x^2$ and $g(x)=1$. To
plot both, we put them both into a list that is enclosed in square
brackets, with the two functions separated by a comma. This produces
graphs of both $f(x)$ and $g(x)$, in two different colors.

How many times do the graphs of $f(x)$ and $g(x)$ intersect?

\subsection{Practice Exercises}

\begin{enumerate}
	\item Graph the functions $f(x)=x^2$ and $g(x)=4$ together
		on the same set of axes. How many times do the
		graphs of $f(x)$ and $g(x)$ intersect?
	\item Repeat with $f(x)=x^2$ and $g(x)=3$. How many times do the
		graphs of $f(x)$ and $g(x)$ intersect?
	\item Repeat with $f(x)=x^2$ and $g(x)=0$. How many times do the
		graphs of $f(x)$ and $g(x)$ intersect?
	\item Repeat with $f(x)=x^2$ and $g(x)=-3$. How many times do the
		graphs of $f(x)$ and $g(x)$ intersect?
	\item What is the relationship between the number of times
		that the graphs of $f(x)=x^2$ and $g(x)=k$ intersect and the
		number of solutions to $x^2=k$?
	\item Are there any intersections of the graphs $f(x) = x^2 + 1$ and $g(x) = x+2$ ? \newline
	Use a sage \verb+plot()+ command to visualize the situation, then use the \verb+solve()+ command to find the points of intersection. 
	\item Notice that in the last problem the \verb+solve()+ command only gives us the $x$ coordinates of the points of intersection.  How can we find the $y$ coordinates?
    \item Recall that the quadratic formula is 

    \[  \frac{-b \pm \sqrt{b^2 - 4ac}}{2a}. \]

    What are the $a$, $b$ and $c$ in this case?  Careful! You need to rearrange so that you have a quadratic polynomial set equal to zero.

    Does the answer from using Sage's \verb+solve()+ command agree with what the quadratic formula tells us?
\end{enumerate}

{\bf \Large Numerical Approximations to Solutions of Equations}

(There are no problems in this section, so just read along and do the computations as you encounter them.)

Some equations are difficult to solve exactly even with the
assistant of a computer and computer software, and Sage is no
exception to this. 

Use the \textbf{solve} function to have Sage solve the
equation $x^5-3x^4+x^3+2=0$. 

The output that we get is
\begin{verbatim}
[0 == x^5 - 3*x^4 + x^3 + 2]
\end{verbatim}

which is another way of Sage telling us that it could not find a solution.
However, plotting the function $x^5-3x^4+x^3+2$ tells us another story.
Plot the graph of $x^5-3x^4+x^3+2$ in Sage to see whether it has
any roots. How many are there, and what are the approximate values
from the graph? You may want to play around with the range of the
$x$-axis (using xmin and xmax) to get a clearer picture.

(3 roots, roughly -0.8, 1.2, and 2.5)

We see that there are 3 roots, at roughly $x=-0.8$, $1.2$, and $2.5$.
Although Sage cannot get the exact values for them by solving
the equation $x^5-3x^4+x^3+2=0$, it can get approximate decimal
values for the roots by using the \textbf{find\_root} function. In
short, a computer algebra system such as Sage uses a sophisticated
version of "find where the graph crosses the $x$-axis, and zoom in
repeatedly around that point to get a more precise estimate of the
$x$-coordinate of where the graph crosses the $x$-axis.'' To find
a root $x$ where $-2<x<0$, we would use the command

\begin{verbatim}
find_root(x^5-3*x^4+x^3+2==0, -2, 0)
\end{verbatim}

The output of this is the root

\begin{verbatim}
-0.7931397744702121
\end{verbatim}

If we want to find a different root, we can change the interval on
which we instruct Sage to look for a root. For example, if we want
to find the root near $1.2$, we could try something like

\begin{verbatim}
find_root(x^5-3*x^4+x^3+2==0, 1, 2)
\end{verbatim}

which gives us the value

\begin{verbatim}
1.199258801379252
\end{verbatim}

Try to find the approximate value of the root of $x^5-3x^4+x^3+2$
near $x=2.5$.

(value is 2.563623765649018)

Note that we said that these values are approximate. Let's verify
what we mean by that. Recall that a root of a function $f(x)$ is
a value of $x$ that makes $f(x)$ equal to exactly $0$, i.e.
$f(x)=0$. Use Sage to define the function $f(x)=x^5-3x^4+x^3+2$,
then plug in the values of the approximate roots from above,
e.g. find $f(-0.7931397744702121)$. If $-0.7931397744702121$
is truly a solution to $x^5-3x^4+x^3+2=0$, then
$f(-0.7931397744702121)$ should equal exactly 0.

However, the value that we get from Sage is

\begin{verbatim}
1.35419453428653e-13
\end{verbatim}

The $e$ here is used for \underline{e}ngineering notation,
where 1.35419453428653e-13 means
\begin{equation*}
1.35419453428653\times 10^{-13}
\end{equation*}

In other words, the part before the ``e'' is a decimal number,
but then the part after the ``e'' is the exponent that we should
raise $10$ to then multiply the decimal. Another way to think of
this is that the ``e-13'' means we should start with the
1.35419453428653 then move the decimal to the left 13 times,
making our number smaller. In contrast, if we had seen

\begin{verbatim}1.35419453428653e5\end{verbatim}

that would mean move the decimal to the right 5 places, resulting
in the number $135419.453428653$.

The number $1.35419453428653\times 10^{-13}$ is a really
small number, but it's not exactly equal to 0. This 
demonstrates that the decimal ``solution'' we obtained using
\textbf{find\_root} is only approximate and not an exact solution.

Another thing to be cautious about when using \textbf{find\_root}
is that it will quit as soon as it finds one approximate root. In other
words, even if there is another root in the interval that you specify,
\textbf{find\_root} will only tell you the value of one of them. If we
try

\begin{verbatim}
find_root(x^5-3*x^4+x^3+2==0, 1, 3)
\end{verbatim}

in hopes of finding both roots that are between 1 and 3, it will not
give us both. Try this out and see what you get!

(only the solution 2.563623765649033 is found)
