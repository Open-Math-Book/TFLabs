%\documentclass[12pt]{amsart}

%\usepackage{amsmath,amssymb,amsthm,amscd,mathtools}
%\usepackage{graphicx,enumitem,color}
%\usepackage[margin=1in]{geometry}
%\usepackage{hyperref}
%\usepackage[font=small,labelfont=bf]{caption}
%\usepackage{subcaption}
%\usepackage{tikz}

%\tikzset{dot/.style={draw,shape=circle,fill=black,scale=.2}}

%\makeatletter
%\newcommand{\newreptheorem}[2]{\newtheorem*{rep@#1}{\rep@title}\newenvironment{rep#1}[1]{\def\rep@title{#2 \ref{##1}}\begin{rep@#1}}{\end{rep@#1}}}
%\makeatother

%\renewcommand\Re{\operatorname{Re}}
%\renewcommand\Im{\operatorname{Im}}
%\newcommand\Int{\operatorname{Int}}
%\newcommand\codim{\operatorname{codim}}
%\newcommand\Res{\operatorname{Res}}
%\newcommand\Arg{\operatorname{Arg}}
%\newcommand\Log{\operatorname{Log}}
%\newcommand\GL{\operatorname{GL}}
%\newcommand\SL{\operatorname{SL}}
%\newcommand\tr{\operatorname{tr}}
%\newcommand\PSL{\operatorname{PSL}}
%\newcommand\Ann{\operatorname{Ann}}
%\newcommand\Spec{\operatorname{Spec}}

%\DeclareMathOperator{\sech}{sech}
%\DeclareMathOperator{\csch}{csch}
%\DeclareMathOperator{\arcsec}{arcsec}
%\DeclareMathOperator{\arccot}{arcCot}
%\DeclareMathOperator{\arccsc}{arcCsc}
%\DeclareMathOperator{\arccosh}{arcCosh}
%\DeclareMathOperator{\arcsinh}{arcsinh}
%\DeclareMathOperator{\arctanh}{arctanh}
%\DeclareMathOperator{\arcsech}{arcsech}
%\DeclareMathOperator{\arccsch}{arcCsch}
%\DeclareMathOperator{\arccoth}{arcCoth} 

%\newtheorem{thm}{Theorem}[section]
%\newtheorem*{thm*}{Theorem}
%\newreptheorem{thm}{Theorem}
%\newtheorem{lem}[thm]{Lemma}
%\newtheorem*{lem*}{Lemma}
%\newtheorem{prop}[thm]{Proposition}
%\newtheorem*{prop*}{Proposition}
%\newtheorem{cor}[thm]{Corollary}
%\theoremstyle{definition}
%\newtheorem{defn}[thm]{Definition}
%\theoremstyle{definition}
%\newtheorem{ex}[thm]{Example}
%\newtheorem{ques}[thm]{Question}

%\newenvironment{modenumerate}
%  {\enumerate\setupmodenumerate}
%  {\endenumerate}

%\newif\ifmoditem
%\newcommand{\setupmodenumerate}{%
%  \global\moditemfalse
%  \let\origmakelabel\makelabel
%  \def\moditem##1{\global\moditemtrue\def\mesymbol{##1}\item}%
%  \def\makelabel##1{%
%    \origmakelabel{##1\ifmoditem\rlap{\mesymbol}\fi\enspace}%
%    \global\moditemfalse}%
%}

%\newcommand{\topfork}{\mathrel{\text{\tpitchfork}}}
%\makeatletter
%\newcommand{\tpitchfork}{%
%  \vbox{
%    \baselineskip\z@skip
%    \lineskip-.52ex
%    \lineskiplimit\maxdimen
%    \m@th
%    \ialign{##\crcr\hidewidth\smash{$-$}\hidewidth\crcr$\pitchfork$\crcr}
%  }%
%}
%\makeatother

\chapter{Symbolic Computer Algebra}


\section{Overview}

We will investigate the relationship between solutions to equations
involving quadratic functions and their graphs by using a computer
algebra system. In doing so, we will learn about the following:
\begin{enumerate}
	\item Making exact computations
	\item Simplifying algebraic expressions
	\item Solving equations
	\item Graphing functions
\end{enumerate}

\section{Computations using computer algebra}

A \textit{computer algebra system} (or \textit{CAS}, for short) is software
that is able to perform algebraic procedures similar to what you would
normally do ``by hand.'' There are several popular CAS used in mathematics,
with the most common being Mathematica, Sage, and Maple. We will
be using Sage, as it is available as free open-source software, and it has been
largely developed by mathematicians, integrating many specialized packages
used by casual and professional mathematics researchers. Much of it
uses Python, so if you have any knowledge of coding in Python, you will find
Sage to be familiar; but fear not if you do not have any prior experience,
as the purpose of this lesson is to get a hang of the basics!

Sage can be installed on your computer or run from the CoCalc servers
if you setup an account. For now, we will use a lightweight version of Sage
that can be run on a webpage without any installation or setting up an
account. Go to \url{https://sagecell.sagemath.org/}.

You can enter basic Sage commands into the textbox on the page and
click ``Evaluate'' to get the output. We will begin by testing out some basic
arithmetic operations.

First enter the following and click ``Evaluate''.

\begin{verbatim}
1+2+3
\end{verbatim}

Sage gives the answer of

\begin{verbatim}6\end{verbatim}

which is the sum $1+2+3=6$. Note that $6$ is interesting in that it is the
sum of all the whole numbers smaller than itself that divide into it evenly
without leaving a remainder (these are referred to as the
\textit{proper divisors} of 6).

We can see that $1$, $2$, and $3$ are proper divisors of 6 without using
a calculator, but we can verify this by calculating $6 \div 1$, $6 \div 2$,
and $6 \div 3$, respectively. For division in Sage (and indeed in most
CAS), we use the ``/'' sign. Try entering

\begin{verbatim}
6 / 3
\end{verbatim}

in Sage and click ``Evaluate''. We will receive back the answer of $2$,
which is of course the value of $6 \div 3$. This is, of course, overkill
for something as simple as $6 \div 3$, but it can be useful when working
with larger numbers or more complicated expressions. For example,
if we want to know whether $9$ divides into $6057$, we would want
to check the value of $6057 \div 9$ and see whether the answer is
a whole number. If it is, then $9$ divides into $6057$ evenly. If it is not
a whole number, then that means $9$ does not divide into $6057$ evenly.
Try calculating $6057 \div 9$ in Sage to see what the answer is.

To verify that what Sage returned is in fact the answer to $6057 \div 9$,
we can take that number and multiply it by $9$ to see if we obtain
$6057$. To do multiplication, we use the ``*'' symbol. For example,
to calculate $2 \times 3$, we would enter

\begin{verbatim}
2 * 3
\end{verbatim}

and click ``Evaluate''. The result is $6$, which is the answer fo $2 \times 3$.
Take the answer you obtained in Sage for $6057 \div 9$ and multiply that
number by $9$ to confirm that you indeed obtain $6057$.

\begin{verbatim}
673 * 9
\end{verbatim}

Now check whether $9$ divides evenly into $70343$ by calculating
$70343 \div 9$ in Sage. What do you get when you evaluate?

The answer that Sage returns is :

\begin{verbatim}
70343/9
\end{verbatim}

From this, it would be tempting to think that Sage is broken because it
did nothing more than return input that you entered as the output.
However, remember that a CAS is a tool designed for doing mathematics,
and in mathematics, we usually prefer \textbf{exact answers} as opposed
to decimals, which are not exact for most fractions (e.g. 0.333333333 is
\textit{not} the same as $\frac{1}{3}$). However, if we want a
decimal approximation for a fraction (or any expression) in Sage, we can
put the whole expression inside of ``N()''. Try evaluating the following in
Sage:

\begin{verbatim}
N(70343/9)
\end{verbatim}

Now we obtain an decimal answer that tells us that $9$ goes into $70343$
times with some remainder. There are several ways to compute the
remainder, but the easiest is to use ``\%''. Try inputting and evaluating the
following:

\begin{verbatim}
70343 % 9
\end{verbatim}

We obtain an answer of 8, which means that the remainder when dividing
70343 by $9$ is 8. We can check that

\begin{verbatim}
8+9*7815
\end{verbatim}

is equal to $70343$, which means $9$ divides into 70343 a total of
7815 times with an additional remainder of 8. Notice another thing
about input and result, which is that unlike a traditional calculator,
when we input a particular expression, Sage will follow the correct
order of operations. That is exponentials will be calculated first,
followed by multiplications and divisions, then additions and subtractions.
Expressions that are inside parentheses will precede anything not
in parenthesis, with the parts in parentheses following orders of
operations as well. If we wanted to do $8+9$ first, then multiply the
result by 7815, we would need to use parentheses, just like in
regular written math

\begin{verbatim}
(8+9)*7815
\end{verbatim}

Finally, the two operations we have not yet discussed are
subtraction  and exponentiation. Two
subtract a number from another number, we use the minus
(``--'') sign. Try

\begin{verbatim}
-317-614
\end{verbatim}

For exponentiating, we have two options:  the caret (`` \^{} '') or 
two consecutive asterisks (``**''). The first is common with many
CAS and is handy when typing on a computer with a keyboard.
The asterisks may be helpful as a shortcut if you are working
on a mobile device where special symbols may be hard to access.

We can see that

\begin{verbatim}
2^3
\end{verbatim}

and

\begin{verbatim}
2**3
\end{verbatim}

both provide the answer to $2^3$.

\subsection{Practice Exercises}

\begin{enumerate}
	\item Calculate $2^0$, $2^0+2^1$, $2^0+2^1+2^2$, 
		and $2^0+2^1+2^2+2^3$. Continue adding the next
		largest power of $2$ until you notice a pattern in the
		result. What is the pattern?
	\item Use Sage to determine all of the proper divisors of 28.
		Then verify that the sum of all of the proper divisors of
		28 is 28.
\end{enumerate}

\subsection{Signing up for CoCalc}

For more advanced calculations that require multiple
steps, it will be beneficial to signup for CoCalc. Follow the following
steps to signup and load a Sage notebook for calculations.

\begin{enumerate}
	\item Go to \url{https://cocalc.com/} in a web browser.
	\item At the top right of the screen, click on the ``Sign Up''
		link.
	\item Follow the sign up instructions and complete
		sign up.
	\item You should now see a page that says ``Signed in as
		XXXX XXXX'' at the top of the page.
	\item Click on the ``Projects'' link at the top left of the page.
	\item In the textbox that says, ``Project title -- you can easily change
		this at any time!'' enter a name for your new project, then
		click on the ``Create New Project'' button.
	\item On the next page, click the ``New'' link near the top
		of the page.
	\item Pick a name for your file, select ``Sage worksheet,''
		then click on the ``Start project'' button at the top of the page.
	\item You will now see a blank file with line numbers on the left.
		By clicking on text area, you will see a blinking cursor where you
		can enter a calculation you would like for Sage to compute.
		After you type in your desired command, click the ``Run'' button
		at the top of the page or type SHIFT$+$ENTER to execute
		that line.
	\item CoCalc will run the calculation on the CoCalc server, so give it
		a few seconds to respond. Once it does, you can continue entering
		new calculations in the lines below the output. Unlike in an
		embedded SageMathCell on a webpage, CoCalc will provide
		output for multiple calculations at once, and it will allow you to
		keep previous calculations on the webpage so that you can
		change them or refer back to them.
\end{enumerate}

\subsection{(optional) Installing Sage}

To install Sage on your own computer, go to
\url{https://doc.sagemath.org/html/en/installation/binary.html} and
download the installer appropriate for your operating system.
While this requires the most setup as you need to install the
software on your computer, calculations will be faster than
running on the CoCalc servers.

\section{Simplifying algebraic expressions}

All of the computations we did in the previous section can be
done on a regular calculator, but where a CAS like Sage really
shines is when working with algebraic expressions including
variables. When working with Sage, we will need to declare
symbolic variables before we use them.

\begin{verbatim}
var('x')
\end{verbatim}

This tells Sage that we will be using the symbol/letter ``x'' as a
unbound variable in an expression. We wish to treat it as symbol
that can represent any possible value and not as a specific
value that is fixed.

\begin{verbatim}
var('x')
3*x+7*x+5
\end{verbatim}

This represents the expression $3x+7x+5$. Note that the implied
multiplication between $3$ and $x$ needs to be specified when
typing into Sage, as does the implied multiplication between
$7$ and $x$. This is because in a CAS, it is also allowable to
have names of variables be words or strings (e.g. ``apple'') and not just
single letters. Thus, $ab$ will be interpreted as the variable called
``ab'' and not the product $a \times b$ of two variables $a$ and $b$
with the multiplication implied by having the two variables next to
each other. Similarly, if you just type in

\begin{verbatim}
var('x')
3x
\end{verbatim}

Sage will not interpret this as the product of $3$ and $x$, so it
will throw an error message of ``invalid syntax'' to tell you that this
input is not valid Sage input. As an example of a variable having
a longer name, consider

\begin{verbatim}
var('x,speed')
x*speed
\end{verbatim}

The first line now declares two variables, one called ``x'' and another
called ``speed''. Then x*speed represents the product of $x$ with the variable
$speed$.

Sage can be helpful when working with algebraic expressions, as
it can do things like expand, factor, or simplify expressions.

If we want to factor an expression like $x^2+4x+3$, we can use
the  \textbf{factor} command.

\begin{verbatim}
var('x')
factor(x^2+4*x+3)
\end{verbatim}

We can also expand out an expression using the \textbf{expand} command.

\begin{verbatim}
var('x')
expand( (x+1)*(x+3) )
\end{verbatim}

It is also often useful to define a function, such as $f(x)$, much like
we do in a regular math class.

\begin{verbatim}
var('x')
f(x)=x^2+4*x+3
\end{verbatim}

We can then evaluate $f(x)$ at different values of $x$ using the standard
notation. For example, to evaluate at $x=1$, we can then enter

\begin{verbatim}
f(1)
\end{verbatim}

which will give us the output of

\begin{verbatim}
8
\end{verbatim}

which is the same as the value of $(1)^2+4(1)+3$, i.e. the value of
substituting $x=1$ into $f(x)=x^2+4x+3$.

We can also easily apply the \textbf{expand} and \textbf{factor}
commands to a function we have already defined.

\begin{verbatim}
f.factor()
\end{verbatim}

will give the output

\begin{verbatim}
x |--> (x+3)*(x+1)
\end{verbatim}

which is the same factorization as when we typed

\begin{verbatim}
factor(x^2+4*x+3)
\end{verbatim}

Similarly,

\begin{verbatim}
f.expand()
\end{verbatim}

can be used to \textbf{expand} the function $f(x)$. In addition,
a function has the \textbf{full\_simplify} option.

\begin{verbatim}
var('x')
f(x)=(x^2+4*x+3)/x+(x+3)*(x-1)/(x+1)
f.full_simplify()
\end{verbatim}

Will simplify the very complicated expression of
\begin{equation*}
	\frac{x^2+4x+3}{x} + \frac{(x+3)(x-1)}{x+1}
\end{equation*}
into a single rational function.

Finally, sometimes Sage will give us an exact expression for
something for which we would like a decimal approximation.
For example, 

\begin{verbatim}
var('x')
f(x)=(x^2+4*x+3)/x+(x+3)*(x-1)/(x+1)
f(2)
\end{verbatim}

gives the output

\begin{verbatim}
55/6
\end{verbatim}

Because Sage is mathematical software, and mathematicians usually
want exact answers, that is what it will return, when possible. In order
to force it to give a decimal expression, we can use the
\textbf{n()} command.

\begin{verbatim}
n(55/6)
\end{verbatim}

gives the decimal approximation of

\begin{verbatim}
9.16666666666667
\end{verbatim}

Alternative, we could have entered

\begin{verbatim}
n(f(2))
\end{verbatim}

instead of 

\begin{verbatim}
f(2)
\end{verbatim}

which would have given us the same result.

\section{Solving equations}

We will investigate functions of the form $f(x)=ax^2+bx+c$. To begin
with, we will consider when $a=1$, $b=0$, and $c=0$, i.e. $f(x)=x^2$.

Start by defining $f(x)=x^2$ in Sage.

\begin{verbatim}
var('x')
f(x)=x^2
\end{verbatim}

To solve the equation $f(x)=1$ (i.e. $x^2=1$) in Sage, we can use the
\textbf{solve} command:

\begin{verbatim}
solve(f(x)==1, x)
\end{verbatim}

This will solve the equation $f(x)=1$, solving for the variable $x$.
Note that when solving, we need to put two equals signs to denote
equality. This is because Sage will interpret a single equals sign
as an assignment (i.e. we would be defining $f(x)$ to be the function
that is always equal to $1$).

The output we obtain is

\begin{verbatim}
[x == -1, x == 1]
\end{verbatim}

These are the two solutions that we expect of $x=-1$ and $x=1$.

\subsection{Practice Exercises}

\begin{enumerate}
	\item Try to solve the equation $f(x)=4$ using Sage.
	\item Solve the equation $f(x)=3$ using Sage.
	\item Solve the equation $f(x)=0$ using Sage.
	\item Do you expect for there to be any solutions for $f(x)=-9$?
		Try it in Sage and see what happens.
	\item For what values of $h$ does $f(x)=h$ have two real solutions?
		Only 1 solution? Two imaginary solutions?
\end{enumerate}

\section{Graphing functions}

Another useful way to visualize functions is by plotting the graph of
functions. Let's begin by plotting the function $f(x)=x^2$:

\begin{verbatim}
var('x')
f(x)=x^2
plot(f(x))
\end{verbatim}

The first two lines define the function $f(x)=x^2$, then the third line will plot
the function. Note that since we did not specify the range for the axes, the
plot will automatically pick a range. If we want to see more of the graph, we
can try adjusting the \textbf{xmin}, \textbf{xmax}, \textbf{ymin}, and \textbf{ymax}
values, which correspond to the minimum and maximum values of the
$x$- and $y$- axes that we desire. For example, changing the plot
command to

\begin{verbatim}
plot(f(x),xmin=-3,xmax=3,ymin=-2,ymax=9)
\end{verbatim}

will gives us an $x$-axis that ranges from $-3$ to $3$ and a $y$-axis that
ranges from $-2$ to $9$.

We can also plot multiple functions on the same plot, in case we
want to compare them together. Let's try overlaying the graph of
the function $g(x)=1$ onto the same set of axes:

\begin{verbatim}
var('x')
f(x)=x^2
g(x)=1
plot([f(x), g(x)],xmin=-3,xmax=3,ymin=-2,ymax=9)
\end{verbatim}

We have defined two functions now, $f(x)=x^2$ and $g(x)=1$. To
plot both, we put them both into a list that is enclosed in square
brackets, with the two functions separated by a comma. This produces
graphs of both $f(x)$ and $g(x)$, in two different colors.

How many times do the graphs of $f(x)$ and $g(x)$ intersect?

\subsection{Practice Exercises}

\begin{enumerate}
	\item Graph the functions $f(x)=x^2$ and $g(x)=4$ together
		on the same set of axes. How many times do the
		graphs of $f(x)$ and $g(x)$ intersect?
	\item Repeat with $f(x)=x^2$ and $g(x)=3$. How many times do the
		graphs of $f(x)$ and $g(x)$ intersect?
	\item Repeat with $f(x)=x^2$ and $g(x)=0$. How many times do the
		graphs of $f(x)$ and $g(x)$ intersect?
	\item Repeat with $f(x)=x^2$ and $g(x)=-9$. How many times do the
		graphs of $f(x)$ and $g(x)$ intersect?
	\item What is the relationship between the number of times
		that the graphs of $f(x)=x^2$ and $g(x)=k$ intersect and the
		number of solutions to $x^2=k$?
\end{enumerate}

\section{Numerical Approximations to Solutions of Equations}

Some equations are difficult to solve exactly even with the
assistant of a computer and computer software, and Sage is no
exception to this. 

Use the \textbf{solve} function to have Sage solve the
equation $x^5-3x^4+x^3+2=0$. 

The output that we get is
\begin{verbatim}
[0 == x^5 - 3*x^4 + x^3 + 2]
\end{verbatim}

which is another way of Sage telling us that it could not find a solution.
However, plotting the function $x^5-3x^4+x^3+2$ tells us another story.
Plot the graph of $x^5-3x^4+x^3+2$ in Sage to see whether it has
any roots. How many are there, and what are the approximate values
from the graph? You may want to play around with the range of the
$x$-axis (using xmin and xmax) to get a clearer picture.

(3 roots, roughly -0.8, 1.2, and 2.5)

We see that there are 3 roots, at roughly $x=-0.8$, $1.2$, and $2.5$.
Although Sage cannot get the exact values for them by solving
the equation $x^5-3x^4+x^3+2=0$, it can get approximate decimal
values for the roots by using the \textbf{find\_root} function. In
short, a computer algebra system such as Sage uses a sophisticated
version of "find where the graph crosses the $x$-axis, and zoom in
repeatedly around that point to get a more precise estimate of the
$x$-coordinate of where the graph crosses the $x$-axis.'' To find
a root $x$ where $-2<x<0$, we would use the command

\begin{verbatim}
find_root(x^5-3*x^4+x^3+2==0, -2, 0)
\end{verbatim}

The output of this is the root

\begin{verbatim}
-0.7931397744702121
\end{verbatim}

If we want to find a different root, we can change the interval on
which we instruct Sage to look for a root. For example, if we want
to find the root near $1.2$, we could try something like

\begin{verbatim}
find_root(x^5-3*x^4+x^3+2==0, 1, 2)
\end{verbatim}

which gives us the value

\begin{verbatim}
1.199258801379252
\end{verbatim}

Try to find the approximate value of the root of $x^5-3x^4+x^3+2$
near $x=2.5$.

(value is 2.563623765649018)

Note that we said that these values are approximate. Let's verify
what we mean by that. Recall that a root of a function $f(x)$ is
a value of $x$ that makes $f(x)$ equal to exactly $0$, i.e.
$f(x)=0$. Use Sage to define the function $f(x)=x^5-3x^4+x^3+2$,
then plug in the values of the approximate roots from above,
e.g. find $f(-0.7931397744702121)$. If $-0.7931397744702121$
is truly a solution to $x^5-3x^4+x^3+2=0$, then
$f(-0.7931397744702121)$ should equal exactly 0.

However, the value that we get from Sage is

\begin{verbatim}
1.35419453428653e-13
\end{verbatim}

The $e$ here is used for \underline{e}ngineering notation,
where 1.35419453428653e-13 means
\begin{equation*}
1.35419453428653\times 10^{-13}
\end{equation*}

In other words, the part before the ``e'' is a decimal number,
but then the part after the ``e'' is the exponent that we should
raise $10$ to then multiply the decimal. Another way to think of
this is that the ``e-13'' means we should start with the
1.35419453428653 then move the decimal to the left 13 times,
making our number smaller. In contrast, if we had seen

\begin{verbatim}1.35419453428653e5\end{verbatim}

that would mean move the decimal to the right 5 places, resulting
in the number $135419.453428653$.

The number $1.35419453428653\times 10^{-13}$ is a really
small number, but it's not exactly equal to 0. This 
demonstrates that the decimal ``solution'' we obtained using
\textbf{find\_root} is only approximate and not an exact solution.

Another thing to be cautious about when using \textbf{find\_root}
is that it will quit as soon as it finds one approximate root. In other
words, even if there is another root in the interval that you specify,
\textbf{find\_root} will only tell you the value of one of them. If we
try

\begin{verbatim}
find_root(x^5-3*x^4+x^3+2==0, 1, 3)
\end{verbatim}

in hopes of finding both roots that are between 1 and 3, it will not
give us both. Try this out and see what you get!

(only the solution 2.563623765649033 is found)

\section{Iterating functions and chaos}

We now consider what happens when applying the function $g(x)=-rx^2+rx$
to a number repeatedly.

We will start when $r=0.5$, so that $g(x)=-0.5x^2+0.5x$. We will start with
the value $x=2$. What is the value of $g(2)$? $g(g(2))$? $g(g(g(2))$?
If we continue to apply $g$ more and more times to the result, what eventually
happens to the output? Use Sage to make your calculations. You may want to
apply \textbf{n()} to your output to get a decimal value to more easily
see the pattern.

Now change $x$ to any number of your choice and repeat the experiment of
applying $g$ repeatedly. What is the eventual outcome?

(all initial values converge to $0$)

Now consider when $r=1.2$, i.e. $g(x)=-1.2x^2+1.2x$.
Start with an initial value of your choice and repeat
the experiment of applying $g$ repeatedly. What is the eventual outcome?

(all initial values converge to $\frac{r-1}{r}=0.1\bar{6}$)

To track the results of iteratively applying the function $g(x)$ to an initial
value of $x=a$, we can create a list of values.

\begin{verbatim}
var('x')
g(x)=-0.5*x^2+0.5*x

a=0.5

points = [a, g(a), g(g(a)), g(g(g(a))), g(g(g(g(a)))), g(g(g(g(g(a)))))]
\end{verbatim}

We can then plot this using the \textbf{list\_plots} function.

\begin{verbatim}
list_plot(points)
\end{verbatim}

If we want to change the value of $a$, we can easily do so, then re-run
the commands that come after it. In addition, if we want to see what
happens when we apply $g(x)$ more times, we can also add those
extra iterates to the list to get a better visualization of what happens
to the output as we apply $g(x)$ more and more times.

Play around with different values of $r$ between $2$ and $4$ and any
choice of the initial value $x$. What patterns do you notice?

Now let's fix a value of $r=3.5$, so we are looking at the function
$g(x)=-3.5x^2+3.5x$. Pick two different values of $x$ that are close to each other 
(within $0.1$ of each other) and plot repeated iterates of $g(x)$ on separate
graphs. What do you notice?
What happens as the two values of $x$ get closer and closer together?

We note that the values seem to be clustered around 4 different values,
and that the iterates seem to cycle through the values systematically. This is called
a \textit{periodic orbit}, and in particular, this is a periodic orbit of length 4 because
the values (approximately) repeat every 4 times that we apply $g$. To find
the periodic points, we can solve for when $x=g(g(g(g(x))))$. Find the points
in this periodic orbit by solving $x=g(g(g(g(x))))$ using Sage.

(Solutions are $x=0, \frac{3}{7}, \frac{5}{7}, \frac{6}{7}$)

