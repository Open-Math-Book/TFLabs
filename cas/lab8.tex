
Most of the computations we did in the previous section can be
done on a regular calculator, but where a CAS like Sage really
shines is when working with algebraic expressions including
variables. When working with Sage, we will need to declare
symbolic variables before we use them.

\begin{verbatim}
x = var('x')
\end{verbatim}

This tells Sage that we will be using the symbol/letter ``x'' as an
unbound variable in an expression. We wish to treat it as symbol
that can represent any possible value and not as a specific
value that is fixed.  In other words, $x$ is a mathematical variable, not a computer variable.

\begin{verbatim}
x = var('x')
3*x+7*x+5
\end{verbatim}

This represents the expression $3x+7x+5$. Note that the implied
multiplication between $3$ and $x$ needs to be specified when
typing into Sage.

Inputting

\begin{verbatim}
var('x')
3x
\end{verbatim}

\noindent will just cause a NameError.

Sage can be helpful when working with algebraic expressions, as
it can do things like expand, factor, or simplify expressions.

If we want to factor an expression like $x^2+4x+3$, we can use
the  {\tt factor} command.

\begin{verbatim}
x=var('x')
factor(x^2+4*x+3)
\end{verbatim}

We can also expand out an expression using the {\tt expand} command.

\begin{verbatim}
x=var('x')
expand( (x+1)*(x+3) )
\end{verbatim}

It is also often useful to define a function, such as $f(x)$, much like
we do in a regular math class.

\begin{verbatim}
x=var('x')
f(x)=x^2+4*x+3
\end{verbatim}

We can then evaluate $f(x)$ at different values of $x$ using the standard
notation. For example, to evaluate at $x=1$, we can then enter

\begin{verbatim}
f(1)
\end{verbatim}

which will give us the output of

\begin{verbatim}
8
\end{verbatim}

which is the same as the value of $(1)^2+4(1)+3$, i.e. the value of
substituting $x=1$ into $f(x)=x^2+4x+3$.

We can also easily apply the {\tt expand} and {\tt factor}
commands to a function we have already defined.

\begin{verbatim}
f.factor()
\end{verbatim}

will give the output

\begin{verbatim}
x |--> (x+3)*(x+1)
\end{verbatim}

which is the same factorization as when we typed

\begin{verbatim}
factor(x^2+4*x+3)
\end{verbatim}

Similarly,

\begin{verbatim}
f.expand()
\end{verbatim}

can be used to {\tt expand} the function $f(x)$. In addition,
a function has the {\tt full\_simplify} option.

\begin{verbatim}
var('x')
f(x)=(x^2+4*x+3)/x+(x+3)*(x-1)/(x+1)
f.full_simplify()
\end{verbatim}

Will simplify the very complicated expression of
\begin{equation*}
	\frac{x^2+4x+3}{x} + \frac{(x+3)(x-1)}{x+1}
\end{equation*}
into a single rational function.

Finally, sometimes Sage will give us an exact expression for
something for which we would like a decimal approximation.
For example, 

\begin{verbatim}
var('x')
f(x)=(x^2+4*x+3)/x+(x+3)*(x-1)/(x+1)
f(2)
\end{verbatim}

gives the output

\begin{verbatim}
55/6
\end{verbatim}

Because Sage is mathematical software, and mathematicians usually
want exact answers, that is what it will return, when possible. In order
to force it to give a decimal expression, we can use the
{\tt n()} command.

\begin{verbatim}
n(55/6)
\end{verbatim}

gives the decimal approximation of

\begin{verbatim}
9.16666666666667
\end{verbatim}

