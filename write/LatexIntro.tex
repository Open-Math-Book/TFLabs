%% LyX 2.3.4.2 created this file.  For more info, see http://www.lyx.org/.
%% Do not edit unless you really know what you are doing.
\documentclass[12pt,english]{article}
\usepackage[T1]{fontenc}
\usepackage[latin9]{inputenc}
\usepackage{geometry}
\geometry{verbose,tmargin=1in,bmargin=1in,lmargin=1in,rmargin=1in}
\usepackage{babel}
\usepackage{graphicx}
\usepackage[unicode=true]
 {hyperref}

\makeatletter

%%%%%%%%%%%%%%%%%%%%%%%%%%%%%% LyX specific LaTeX commands.
%% Special footnote code from the package 'stblftnt.sty'
%% Author: Robin Fairbairns -- Last revised Dec 13 1996
\let\SF@@footnote\footnote
\def\footnote{\ifx\protect\@typeset@protect
    \expandafter\SF@@footnote
  \else
    \expandafter\SF@gobble@opt
  \fi
}
\expandafter\def\csname SF@gobble@opt \endcsname{\@ifnextchar[%]
  \SF@gobble@twobracket
  \@gobble
}
\edef\SF@gobble@opt{\noexpand\protect
  \expandafter\noexpand\csname SF@gobble@opt \endcsname}
\def\SF@gobble@twobracket[#1]#2{}

%%%%%%%%%%%%%%%%%%%%%%%%%%%%%% Textclass specific LaTeX commands.
\newenvironment{lyxcode}
	{\par\begin{list}{}{
		\setlength{\rightmargin}{\leftmargin}
		\setlength{\listparindent}{0pt}% needed for AMS classes
		\raggedright
		\setlength{\itemsep}{0pt}
		\setlength{\parsep}{0pt}
		\normalfont\ttfamily}%
	 \item[]}
	{\end{list}}

\makeatother

\begin{document}

\section*{Introduction to \LaTeX\protect\footnote{Contributed by Len Brin, November 2021}}

\subsection*{The Web and Markup}

Have you ever seen a web URL like 
\begin{center}
\texttt{\href{https://www.agnesscott.edu/lriddle/women/love.htm}{https://www.agnesscott.edu/lriddle/women/love.htm}}
\par\end{center}

\noindent and wondered what the \texttt{https} or the \texttt{htm}
meant? In both \texttt{https} and \texttt{htm}, the \texttt{ht} is
short for hypertext. The \texttt{tp} in \texttt{https} is short for
transfer protocol and the \texttt{s} is for secure. The \texttt{m}
in \texttt{htm} is for markup. All webpages, when it comes down to
it, are made of HTML (hypertext markup language) code. The HTML tells
the browser what should be displayed and, in general terms, the desired
layout. The browser takes care of the fine details depending on the
type of device, size of the screen, or personal preferences set by
the user.

Markup in HTML is like a note to the browser telling the browser how
to display the following content. The markup won't be displayed, but
it will modify how the content will be displayed. Take this screenshot
from \texttt{\href{https://www.southernct.edu/}{https://www.southernct.edu/}}
for instance.
\begin{center}
\includegraphics[width=4in]{SCSUscreenshot}
\par\end{center}

\noindent Simplified just a bit, the HTML code that produces the ``Back
to Campus'' announcement above looks like this:
\begin{verbatim}
<img src="/sites/default/files/back-to-campus-banner.jpg" /><p style
="font-size: 18px;">As the university prepares for the start of the
fall 2021 semester, we remain #SouthernStrong, with <a href="https://
inside.southernct.edu/reopening#services">all of the following academic
offerings and campus services</a> in place for our students in a safe,
engaging campus environment. Our goal is to get you off to a great
start, whether you're returning to campus or joining our community for
the first time!</p>
\end{verbatim}
It is just simple text! No bells, no whistles. Parts enclosed by \texttt{<}
and \texttt{>} are the markup. These parts will not be displayed themselves
but rather describe how something is to be displayed and leaves the
rest up to the browser. For example,
\begin{center}
\texttt{<img src=\textquotedbl /sites/default/files/back-to-campus-banner.jpg\textquotedbl{}
/>}
\par\end{center}

\noindent tells the browswer to insert an image (\texttt{img}) and
tells the browser where to get the image (\texttt{src}). This part
of the code creates
\begin{center}
\includegraphics[width=4in]{back-to-campus-banner}
\par\end{center}

\noindent The rest of the code creates the paragraph underneath this
image. It starts with \texttt{<p style=\textquotedbl font-size: 18px;\textquotedbl >}---which
tells the browser to start a paragraph (\texttt{p}) using an 18 pixel
font---and ends with \texttt{</p>}, which tells the browser this
is the end of the paragraph. In the middle of the paragraph is what
makes webpages webpages---a hyperlink. The text between \texttt{<a
href=\textquotedbl https://inside.southernct.edu/reopening\#services\textquotedbl >}
and \texttt{</a>} is marked up as a link to \texttt{https://inside.southernct.edu/reopening\#services}.
The browser knows to display it in blue so the reader can see that
these words form a link. Clicking it will bring up a new webpage in
the browser.

\subsection*{Write Your Own Markup}

To see markup in action, point your browser to \texttt{\href{https://htmledit.squarefree.com/}{https://htmledit.squarefree.com/}},
where you can write some hypertext markup and see how it looks on
your browser. The blue box at the top holds the markup, and it can
be edited by you! The box below shows how the browser renders the
markup. Do the following exercises on the \texttt{squarefree} webpage
and answer the questions.
\begin{enumerate}
\item Insert \texttt{<em>} before the word magically and insert \texttt{</em>}
after the word magically. What did this accomplish? Note: \texttt{em}
is short for \textit{em}phasis!
\item Copy and paste the following code into the webpage.
\begin{lyxcode}
<p>A~list~of~some~common~HTML~markup

<ol>

~<li><tt>p</tt>~is~short~for~<b>p</b>aragraph</li>

~<li><tt>a</tt>~is~short~for~<b>a</b>nchor~(which~can

indicate~a~link~or~a~place~to~link~to)</li>

~<li><tt>ol</tt>~is~short~for~<b>o</b>rdered~<b>l</b>ist</li>

~<li><tt>li</tt>~is~short~for~<b>l</b>ist~<b>i</b>tem</li>

</ol>

Notice~that~markup~can~be~nested-{}-the~b~and~/b~tags~above~are

inside~the~li~and~/li~tags,~which~are~between~the~ol~and~/ol

tags.</p>
\end{lyxcode}
\begin{enumerate}
\item What happens to text between the \texttt{<b>} and \texttt{</b>} tags?
\item What happens to text between the \texttt{<tt>} and \texttt{</tt>}
tags?
\end{enumerate}
\item Now change the \texttt{<ol>} and \texttt{</ol>} tags to \texttt{<ul>}
and \texttt{</ul>} tags. What happens to the displayed page (in the
white box)? Note: \texttt{ul} is short for \textbf{u}nordered \textbf{l}ist.
\end{enumerate}

\subsection*{\LaTeX\  (pronounced lah-tek or lay-tek)}

Markup is used to tell a renderer how you want something to look,
but in a general way that frees you to work on content, not layout.
Markup is a very powerful way to make documents that are
\begin{itemize}
\item readable by humans (and therefore editable by humans) and
\item portable (transferable between devices)
\item with a consistent layout (the author does not worry about the fine
details of the layout)
\end{itemize}
As we have seen, HTML (the markup language of webpages) is used to
control how a webpage appears. \LaTeX\  plays the same role for printed
documents like books, reports, letters, or labs (like this one!).
\LaTeX\  is a markup language that excels at creating technical documents,
especially those that include mathematical formulas. You might want
to read \href{https://medium.com/swlh/the-students-guide-to-latex-markup-what-it-is-and-why-you-want-it-651e723ce0c8}{this blog}
all about why you want \LaTeX.

Like HTML, \LaTeX\  uses tags to mark how text should look, to insert
graphics, to create lists, and so on. Markup in \LaTeX\  begins with
a \textbackslash{} (a backslash) as in \texttt{\textbackslash pi},
which would be used to render our friend, $\pi$. Every \LaTeX\ 
document begins with the \texttt{\textbackslash documentclass} tag,
which specifies what type of document is to be created (book, article,
report, etc.).\footnote{This is similar to HTML documents, which all begin with the \texttt{<html>}
tag and end with the \texttt{</html>} tag.} After the documentclass tag a \LaTeX\  document must include the
tags \texttt{\textbackslash begin\{document\}} and \texttt{\textbackslash end\{document\}}.\footnote{This is similar to HTML documents, in which the beginning of the content
is marked by \texttt{<body>} and the end by \texttt{</body>}.} As you can probably guess, these tags mark the beginning and end
of the content of the document. One of the simplest \LaTeX\  documents
possible is this one:
\begin{lyxcode}
\textbackslash documentclass\{article\}

\textbackslash begin\{document\}

Hello~World!

\textbackslash end\{document\}
\end{lyxcode}
Just like HTML code, it isn't pretty! However, it tell the \LaTeX\ 
renderer what to do. This code will create an ``article'' with the
sentence ``Hello World!'' in it. Try it yourself!
\begin{enumerate}
\item Point your browser to \texttt{\href{https://latexbase.com}{https://latexbase.com}}.
On the left side (black box) is where you put the \LaTeX\  code and
on the right side (white box) is where it shows you your document.
\item Delete all the code.
\item Copy and paste the code from above into the black box at \texttt{latexbase.com}.
\item Wait a moment...
\end{enumerate}
You should see your document on the right side. It has only the words
``Hello World!'' (plus just a touch more) because that's all we
put in the document. Scroll down to the bottom of the page and you
will see the number 1, the page number. Pages of articles are normally
numbered, so \LaTeX\  puts that in for you! If you don't want page
numbers because your whole document is only going to be one page or
you are writing a letter, you can get rid of the page number by inserting
the markup \texttt{\textbackslash pagestyle\{empty\}}. This belongs
between the \texttt{\textbackslash documentclass\{article\}} and
the \texttt{\textbackslash begin\{document\}} tags, an area known
as the \textbf{preamble} of the \LaTeX\  document. Try it!
\begin{enumerate}
\item Insert \texttt{\textbackslash pagestyle\{empty\}} (on its own line)
between \texttt{\textbackslash documentclass\{article\}} and the
\texttt{\textbackslash begin\{document\}}.
\item Wait a moment...
\end{enumerate}
When \texttt{latexbase} indicates that your document is ready (you
will see the \includegraphics[height=12pt]{ready} icon on the right
side toward the top of the page), scroll down to the bottom again.
The page number will be gone.

\subsection*{Adding a list}

Now add a list of your three favorite classes of all time---of course
math class is first on your list, so that one has been put in for
you. You'll have to supply the next two...
\begin{enumerate}
\item Copy and paste the following markup into your document on \texttt{latexbase}
(after ``Hello World!'' but before \texttt{\textbackslash end\{document\}}
so that it is included in the document and appears after ``Hello
World!'')
\begin{lyxcode}
\textbackslash par

My~three~favorite~classes~of~all~time~are

\textbackslash begin\{enumerate\}

~~~\textbackslash item~Math

\textbackslash end\{enumerate\}
\end{lyxcode}
Notice that the \texttt{\textbackslash par} tag does not have a begin
or end. It only marks where a new paragraph should start. Same with
the \texttt{\textbackslash item} tag. It only marks where a new item
in the list should start.
\item Add two items to the enumeration (and you can change the first item
if by some strange chance math is not your all time favorite class).
\end{enumerate}

\subsection*{Adding an equation}

Equations in \LaTeX\ come in two varieties---inline and display.
An inline equation is any mathematical expression that appears in
the middle of a sentence (like the $\pi$ right here and earlier in
this document). A display equation is any mathematical expression
that should appear centered on its own line (because it's super important
or just because it's too big to put in the middle of a sentence).

To put an equation in the middle of a sentence, enclose the math between
two dollar signs (\$). To add a display equation, enclose the math
between double dollar signs (\$\$). Try it!
\begin{enumerate}
\item Copy and paste the following markup into your document on \texttt{latexbase}.
\begin{verbatim}
My favorite mathematical constant is $\pi$, but I like $e$ too.
Did you know that $$e^{i\pi}=-1?$$ Weird...
\end{verbatim}
\item Notice that exponents are typeset using the same notation as used
on a calculator! Can you add markup to your document that will produce
the following:
\begin{quote}
The Pythagorean Theorem states that if a triangle has legs of lengths
$a$ and $b$ and hypotenuse of length $c$, then
\[
a^{2}+b^{2}=c^{2}.
\]
\end{quote}
\end{enumerate}

\subsection*{Homework Exercises}
\begin{enumerate}
\item To come...
\end{enumerate}

\end{document}
