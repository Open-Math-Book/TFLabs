\begin{enumerate}
	\item Did you fix your name in the Overleaf ``Account'' area?
	\item Okay. Then, start a new blank project called ``more''
	\item Add the following lines into the preamble of your document:
\medskip

\begin{codeblock}
\begin{verbatim}
%\usepackage{geometry}
%\usepackage{hyperref}
\end{verbatim}
\end{codeblock}

Of course, the commands you've just entered start with the comment symbol so they won't {\em do} anything.  Did you notice how the Overleaf source editor renders them in green?

	\item Now, add the following to the body of your document.  (After the \verb+\section{Introduction}+ command)
\medskip

\begin{codeblock}
\begin{verbatim}
The first thing you should notice is that the text is a little
bigger than it was last time.  You may also have noticed that 
the margins are a bit big by default.  For illustration purposes 
we need a paragraph that is long enough to ensure we've 
reached the right margin.  This one should do the trick.
\end{verbatim}
\end{codeblock}
\medskip

	\item Hit the ``Recompile'' button.  Did you notice is that the text is a little bigger than it was last time?
	\item Try deleting the \verb+[12pt]+ optional argument to the \verb+\documentclass+ command.  Hit ``Recompile.'' Do you see the difference now?  
	\item Put the \verb+[12pt]+ optional argument back and recompile one last time.
	\item Okay, now what about that comment about \LaTeX{}'s margins?  Try uncommenting the line that includes the {\tt geometry} package.
	Recompile.  What's changed?
	\item The {\tt geometry} package will give you minute control over your document's margins.  The default margins when the {\tt geometry} package is in use are a good bit smaller than \LaTeX{}'s defaults.
	\item Try replacing the {\tt geometry} line with this one:
\medskip

\begin{codeblock}
\begin{verbatim}
\usepackage[bottom=1in, right=.5in, left=.5in, top=1in]{geometry}
\end{verbatim}
\end{codeblock}
\medskip

    \item Play around a bit.  What settings for the margins are most pleasing to your eyes?
    \item Now, uncomment the \verb+\usepackage{hyperref}+ command.  Recompile.
    \item Did you see anything change?
    \item If you did, it's probably your imagination. Hyperref doesn't make any visible changes -- it provide you with some new commands!
    \item Drop the following line into your source file:
\medskip

\begin{codeblock}
\begin{verbatim}
This is a 
\href{https://www.cespedes.org/blog/85/how-to-escape-latex-special-characters}{link} 
to a page that gives some good information about special characters in \LaTeX{}.  
You may notice that the {\tt href} command provided by the {\tt hyperref} package 
is our first example of a \LaTeX{} command that has two arguments.
\end{verbatim}
\end{codeblock}
\medskip

	\item I don't know about you, but I really don't like how hyperlinks are rendered by default.  If you put the following right after the \verb+\usepackage{hyperref}+ command you'll get a nicer appearance.
\medskip

\begin{codeblock}
\begin{verbatim}
\hypersetup{colorlinks=true,  urlcolor=blue}
\end{verbatim}
\end{codeblock}
\medskip


\end{enumerate}
