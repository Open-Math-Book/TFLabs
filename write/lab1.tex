% !TEX root = lab1_.tex
To see markup in action, point your browser to 

\centerline{ \texttt{\href{https://htmledit.squarefree.com/}{https://htmledit.squarefree.com/}}, }

\noindent where you can write some hypertext markup and see how it looks on
your browser. The blue box at the top holds the markup, and it can
be edited by you! The box below shows how the browser renders the
markup. Do the following exercises on the \texttt{squarefree} webpage
and answer the questions.
\begin{enumerate}
\item Insert \texttt{<em>} before the word magically and insert \texttt{</em>}
after the word magically. What did this accomplish? Note: \texttt{em}
is short for \textit{em}phasis!
\item Copy and paste the following code into the webpage.

\begin{verbatim}
<p>A list of some common HTML markup
<ol>
 <li><tt>p</tt> is short for <b>p</b>aragraph</li>
 <li><tt>a</tt> is short for <b>a</b>nchor (which can
indicate a link or a place to link to)</li>
 <li><tt>ol</tt> is short for <b>o</b>rdered <b>l</b>ist</li>
 <li><tt>li</tt> is short for <b>l</b>ist <b>i</b>tem</li>
</ol>

Notice that markup can be nested -- the b and /b tags above are
inside the li and /li tags, which are between the ol and /ol
tags.</p>
\end{verbatim}


\item What happens to text between the \texttt{<b>} and \texttt{</b>} tags?
\item What happens to text between the \texttt{<tt>} and \texttt{</tt>}
tags?


\item Now change the \texttt{<ol>} and \texttt{</ol>} tags to \texttt{<ul>}
and \texttt{</ul>} tags. What happens to the displayed page (in the
white box)? Note: \texttt{ul} is short for \textbf{u}nordered \textbf{l}ist.

Make a new ordered list and use it to store your answers to the lab questions \#1, \#2, etc.

\item Notice how the b and /b tags in the last sentence of the copy/pasted code above are missing the usual angle brackets.  What happens if you put them in?
\item This is a problem in all markup languages -- some characters have special meaning.  In HTML there are so-called {\em escape characters} to work around this issue.  Google the phrase ``html escape characters'' and see if you can re-write the last sentence so that things {\bf look like} a tag, but don't {\bf act like} a tag!
\item Look up how to use the \texttt{<a href="...">...</a>} and \texttt{<img src="..."></img>} tags to make a link to another website and embed an image in your page.
\item Try writing some text and put a single \texttt{<br>} tag somewhere. What happens? Why doesn't it need a ``closing'' tag?
\item Similarly, what does the \texttt{<hr>} tag do?
\item There are lots of different computer programs (``browsers'') that need to translate the html source code of the webpages you visit into the beautiful layouts you see on your screen. How many different browsers can you think of?
\item Who is in charge of deciding what those browsers need to recognize as official html code? (Hint: search the internet for ``W3C.'') Who is in charge of that organization, and where do its profits go?
\end{enumerate}