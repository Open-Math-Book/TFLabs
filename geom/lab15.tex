\begin{enumerate}

\item Start GeoGebra, sign in, and change the mode from ``Graphing'' to ``Geometry''

\vfill

\item When you first enter the ``Geometry'' mode the tool panel will have a very limited number of options.  All but one of the things we'll need for the first task are here -- the additional tool is call ``Intersect'' and you'll need to fully expand the tool panel to find it.  Use points, lines, circles and the ability to construct points at the intersection of two curves to construct an equilateral triangle.

\vfill

\item Create a line and a point that's not on the line.  Can you use the basic tools (constructing lines and circles, finding intersections) to create a line that goes through the point, which is also perpendicular to the original line?

\vfill

\item Once you've completed the previous task, we're justified in letting you use the ``Perpendicular Line'' tool.  Using the basic tools plus ``Intersect'' and ``Perpendicular Line'' construct a square.

\vfill

\item Using the same tools, construct a right triangle where one leg is twice as long as the other.  If we say the short side is one unit long, how long is the hypotenuse of this triangle?

\vfill

\item There is a number you may have heard of called $\phi$ -- the golden proportion.  Its value is $\displaystyle \phi = \frac{1+\sqrt{5}}{2}$.
A golden rectangle is a rectangle whose side are in the golden proportion, so if you can construct a rectangle with sides $2$ and $1+\sqrt{5}$ you will have made a golden rectangle.  Referring to the previous problem, you should be able to do it!

\vfill

\item Let's finish up by exploring some properties of triangles.  Feel free to use any of the tools for these questions.
There are several ways to find a point that can be regarded as the ``center'' of a triangle.  Use Geogebra to construct all of the following:

\begin{enumerate}
	\item The line segments that go from a corner to the midpoint of the edge across from it.  There are three of these, which intersect in a point called the {\em mediant}.
	\item The line segments that are perpendicular to an edge and pass through the point across from the edge.  These intersect in a point called the {\em orthocenter}.
	\item The lines that bisect the angles at each vertex.  These intersect in a point called the {\em incenter}.
\end{enumerate}

\item Create a single triangle and construct the mediant, orthocenter and incenter.  Change the settings so that each type of ``middle'' is constructed with a different color.  Now move the corners around to see if

\begin{enumerate}
	\item it is possible to make all 3 centers coincide.
	\item it is possible to make two centers coincide while the 3rd is different.
\end{enumerate}

\end{enumerate}
