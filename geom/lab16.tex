
\noindent In this lab, you will learn how to use GeoGebra to construct a geometric object and how to apply the distance definition of parabola to analyze your construction. Remember, \textbf{a parabola} is the set of points equidistant from a fixed point (the focus) and a fixed line (the directrix). 
\vspace{0.2 in}

In fact, you can create a parabola with nothing more than a sheet of wax (or plain) paper and a single point on the page! The video in this link will show you how one can fold paper to create a parabola: \url{https://www.youtube.com/watch?v=GdJlbNweSVY}. 

\vspace{0.2 in}

But, creasing your paper takes some work. Folding one or two sheets is fun, but what would happen if you wanted to continue testing many different locations for point A? You’d need to keep starting over with fresh paper, folding new sets of creases. Technology can streamline your work. With just one set of creases, you can drag the Focus point to new locations and watch the crease lines adjust themselves instantaneously!


\begin{enumerate}

\item Before trying this on the computer, try doing it with actual paper!

\vfill

\item Start GeoGebra, sign in, and put it in ``Geometry'' mode.

\vfill

\item Use the Line tool to draw a horizontal line near the bottom of the screen. This line AB represents the bottom edge of the paper (the directrix of the parabola).

\vfill

\item Draw a point C above the line, roughly centered between the left and right edges of the screen.	This point will be the focus of your parabola.

\vfill

\item Construct a point D anywhere on the horizontal line and construct the “crease” formed when point D is folded onto point C. There may be other options, but the ``perpendicular bisector tool'' would be a good choice.

\vfill

\item Drag point D along its line. If you constructed your crease line correctly, it should adjust to the new locations of point D.

\vfill

\newpage

\item Select the crease line and choose ``Show trace'' from the right-click menu.

\vfill

\item Drag point D along the horizontal line to create a collection of crease lines.

\vfill

\item To see other cases, you'll want to clear the traces.  Look in the drawing window settings (the gear in the upper right corner of the drawing area) and select ``Clear all Traces.''  Now try moving the focus (point C) to a different location.

\vfill

\item Again, drag point D to create another collection of crease lines.

\vfill 

\item There is a point that actually traces out the parabolas we've been looking at.  It lies at the intersection of the crease line and a line through D, perpendicular to the directrix.  Construct it!

\vfill

\item To create the graph of your parabola, without all the trace lines, check out the ``Locus'' tool.  It draws the path that a point, whose position is determined by something else, will move in.  The tool tip on this is not terribly clear - click the ``Locus'' tool, then the point whose path you want drawn (the locus point) and then the point that controls its motion (point D in this case).

\vfill

\item You can also create the graph of a parabola using the Algebra panel.  Just type \verb+y = x^2+, and you'll get the parabola most people think of when the hear the word ``parabola.''  Challenge: move the directrix and the focus around until you get a close match.

\vfill

\item Final Challenge: What do the think are the actual locations of the focus and the directrix for $y=x^2$?




\end{enumerate}
