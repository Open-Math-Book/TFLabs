\begin{enumerate}

\item Start GeoGebra, sign in, and change the mode from ``Graphing'' to ``Geometry''

The interface will now have two large panes.  On the right is the drawing area.  On the left is where you can select from a bunch of different tools.

One source of annoyance is that the system does whatever action is dictated by the tool that is currently active.  So, for example, if you've selected the ``Point'' tool and you then try to move the view, you'll create a point that you hadn't really intended\ldots

One day you'll get used to this, but in the meantime there's an ``undo'' button\ldots

They've recently added a shortcut to get back to the ``Move'' cursor (since that's the most frequent switch we'll want to make, this is very handy)
\vfill

\item When you first enter the ``Geometry'' mode the tool panel will have a very limited number of options.  Expand them by clicking ``More.''  That gives you quite a few more possible tools to use, but at the bottom of the panel you'll find another ``More'' button.  After you click that one, every option will be displayed.  Look through the list of tools and figure out how to draw a regular hexagon.

\vfill

\item Use the ``Point'' tool to draw 5 random points in the drawing area.  

\vfill

\item Locate the shortcut for getting back to the ``Move'' cursor and then practice moving some of your points around.

\vfill

\item Fairly far down in the tool panel, you should find something labelled ``Conic through 5 points.''  Use it to create the conic that passes through your 5 points.

\vfill

\item A conic (a.k.a. a conic section) is the sort of curve that is created where a plane intersects with a cone.  There are four basic types: circles, ellipses, parabolas and hyperbolas.  Hyperbolas are the strangest case because they have two disjoint parts.  Try moving your points around so that you get each sort of conic.

\vfill

\item There are two so-called ``degenerate'' hyperbolas that consist of two lines -- they can either be crossing or parallel.  Move some or all of your points to create both degenerate hyperbolas.

\vfill

\item Right clicking on an object allows you to access a wide range of ``Settings'' for it.  It also gives you the option to ``Show trace'' for the object.  Change your conic's color to something nice.  Also, adjust the thickness of the curve to something you like. To get rid of the ``Settings'' panel, look for the X in the upper-right corner.

\vfill

\item Turn on the ``Show trace'' function for your conic, then play around with moving one or more of the 5 points that were used to define it.  The images you'll generate are pretty cool\ldots

\vfill 

\item Right click again on the conic and open its ``Settings.''  Deselect the radio button that is labelled ``Show Object'' and then close the Settings panel.  Notice that now we're in a pretty bad state.  We've just made the conic invisible, but now there's no way to click on it so we can change its settings (for instance to make it visible again!)  This is one of many places where an alternative view that we haven't yet talked about comes in handy.  Look at the extreme left side of the window.  You should see two icons labelled ``Tools'' and ``Algebra.''  So far we've only looked at the ``Tools'' view.  Clicking it over to ``Algebra'' gives another way of looking at the things we've constructed.  In particular, it should be easy now to make the conic visible again.

\vfill

\item Finally, let's do something with that hamburger in the upper left corner.  Look into those menu options and figure out how to export a graphic file of your construction.  Make graphics of all of the different kinds of conics (including the degenerate cases).

\vfill

\end{enumerate}
