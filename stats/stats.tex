 
\chapter{Statistics}


\section{R}
\label{sec:r}

\subsection{Getting access to R}

\subsubsection{R in your Browser}

Log into your CoCalc account and create a new sage worksheet. There is something called ``cell magic'' that changes the nature of a cell so that it ``understands'' other languages besides sage.  For instance, typing \verb+%md+ on the first line of a cell makes that cell a markdown cell.  Typing \verb+%r+ on the first line of a cell makes that cell an R cell.

It's easy to forget to put in the cell magic, which leads to some hilarious (but disconcerting) error messages.  Fortunately, there's a way to set the cell magic globally.
If you put \verb+%default_mode r+ in the first line of the first cell of a worksheet (and execute it!) then all of the cells will be treated as containing R code.

% ~~~~~~~~~~~
\subsubsection{Downloading R and RStudio}

There's no need to do this for today's lab, but if you want to get R running on your own machine\dots

To download R to your computer, you need to download two pieces of software
\begin{enumerate}
\item R, which is like the engine underneath the hood
\item RStudio, which is the controls and the software you'll actually work with.
\end{enumerate}

To begin your download of R, click on the following link

\begin{center}
  \href{http://lib.stat.cmu.edu/R/CRAN/}{http://lib.stat.cmu.edu/R/CRAN/}
\end{center}

Then select the link corresponding to your computer type.

Move forward as if downloading any other program. Once you download R, you still need to download RStudio.

% ~~~~~~~~~~~~~~~~~~~~~~~~~~~~~~~~
\subsubsection{Downloading RStudio}

To begin downloading RStudio, click on the following link

\begin{center}
\href{https://www.rstudio.com/products/rstudio/download/}{https://www.rstudio.com/products/rstudio/download/}
\end{center}

Select the ``Download RStudio Desktop'' button


Then select the download link corresponding with your operating system.


\subsection{Variables}
\label{sec:r_variables}

In the context of R, the term \textbf{variable} is not an unknown
quantity to be determined, as in ``Solve for $x$ in $3x-5=10$'' It
\textbf{is} a nickname, given for convenience.  We will know exactly
what it is, because we define it.  

A slight oddity of the R language is that we don't use an equals sign ($=$) to assign a nickname, instead we type \verb+<-+ which is supposed to look like a left-pointing arrow.

Sometimes, we like to choose a short variable name that is simple to type
  
\begin{example}
  \begin{rcode}    
    a <- 5 # assign the name 'a' to the value '5'
    3*a
    5*a^2+5
    
    b <- 2*a-6
    b
    
    3*a^b-2
  \end{rcode}
\end{example}

Othertimes, it's best to use a variable name that is descriptive so
when we read our code later, we understand what it is doing

\begin{example}[Convert kg to lb's]
  \begin{rcode}    
    weight_kg <- 50 # assign name weight_kg to the number 50
    weight_lb <- 2.2*weight_kg # assign convert weight_kg and assign new name
    weight_lb # print the value of weight_lb
  \end{rcode}
  \textbf{Important:} can't use spaces in variable names, use underscore instead.
\end{example}
  
% ~~~~~~~~~~~~~~~~~~~~~~~~~~~~~~~~~~~~
\subsection{Functions}
\label{sec:r_functions}

We'll learn a bunch of \define{functions} in R.  

\begin{example}[Absolute Value Function]
  The absolute value function is written like ``abs()''.  Place the
  value you want to take the absolute value of into the parenthasis.
  \begin{rcode}
    abs(4) # absolute value of 4
    abs(-3) # absolute value of -3
    
    x <- -3*5 + 2
    abs(x)
    abs(-3*5+2) # same as abs(x)

    abs(y) # notice the error because we haven't told R what 'y' is
  \end{rcode}    
\end{example}

\begin{example}[Rounding function]
  A common function we'll want to use to rounding numbers.  There is
  a function to do this for us.  Using this function is better than
  doing it yourself.
  \begin{rcode}
    number <- 4.3134314512351
    round(number) # round to nearest whole number
    round(number, 2) # round to 2 decimal places
    round(number, 4) # round to 4 decimal places
  \end{rcode}    
\end{example}

It is incredibly common to forget exactly how a function works, or
maybe you're getting an error that you can't quite crack. There are
two ways of getting help.

\begin{enumerate}
\item Google.  If you can't recall how the \texttt{round()} function
  works, google ``round function R''
\item Help within R itself.  There is an R function called \texttt{help()} -- put whatever R command you're interested in in the parentheses.
\item If you're running RStudio, go to the lower right window, click the
  'help' tab, and search for your function.
\end{enumerate}

\subsection{Exercises}

\begin{q}
  What R code is used to compute the following expressions?
  \begin{enumerate}
  \item Subtract 10 from 2 
  \item Multiply 3 by 5
  \item Divide 9 by 2
  \item From 5, subtract 3 times 2
  \item Double the difference of 5 and 3
  \item Four squared
  \item Divide 8 by 2 squared
  \end{enumerate}
\end{q} 

% ~~

\begin{q}
  Consider the following R code.

\begin{rcode}
mass_kg <- 2.62 
mass_g <- mass_kg * 1000 
mass_g 
\end{rcode}

\begin{enumerate}
\item Explain in plain language what the following three line program
  does, i.e. state what each line does and what the meaning of the
  output of the program is
\item  Run the program. (Hit enter after each line break.) What is its output?  
\item  Modify this code so that it converts the inital mass to pounds (there are 2.2
  pounds in a kilogram) and print the resulting mass.  Run this
  program.
\end{enumerate} 
\end{q} 

% ~~

\begin{q} 
  Run \texttt{help(abs)} in R. What happens?  Familiarize yourself
  with the built-in functions \texttt{abs()}, \texttt{round()},
  \texttt{sqrt()} by running this ``help function'' on \texttt{round}
  and \texttt{sqrt}. Use these built-in functions to create R code to
  accomplish the following:

  \begin{enumerate}
  \item The absolute value of -15.5.   
  \item 4.483847 rounded to one decimal place. 
  \item Assign the value of the square root of 2.6 to a variable. Then
    round the variable you’ve created to 2 decimal places and assign
    it to another variable. Print out the rounded value.
  \end{enumerate} 
\end{q}

% ~~

\begin{q}
  The following code estimates the total net primary productivity
  (NPP) per day for two sites. It does this by multiplying the grams
  of carbon produced in a single square meter per day by the total
  area of the site. It then prints the daily NPP for each site.

\begin{rcode}
site1_g_carbon_m2_day <- 5 
site2_g_carbon_m2_day <- 2.3 
site1_area_m2 <- 200 
site2_area_m2 <- 450 
site1_npp_day <- site1_g_carbon_m2_day * site1_area_m2 
site2_npp_day <- site2_g_carbon_m2_day * site2_area_m2 
site1_npp_day 
site2_npp_day 
\end{rcode}

\begin{enumerate}
\item Explain what each line does. You can refer to each line number, e.g. ``Line 1 does so and so''. 
\item Run the program using an RScript (upper left window, highlight
  everything, then click 'Run'). What is the output of the program?
\item Continuing the above code, write a line of code for each of the
  following items
  \begin{enumerate}
  \item The sum of the total daily NPP for the two sites combined.  
  \item The difference between the daily NPP for the two sites. We
    only want an absolute difference, so use abs() function to make
    sure the number is positive.  
  \item The total NPP over a year for the two sites combined (the sum
    of the total daily NPP values multiplied by 365).
  \end{enumerate}  
\end{enumerate}

\end{q} 

% ==================================
\section{Vectors in R}

One of the most fundamental objects we work with in R is known as a vector.  Think of it like a list of values. Often it may come as a column from a spreadsheet.  R gives us ways to perform different mathematical tasks with these vectors.  In this topic, most of the vectors will not contain too many items, so we'll learn what R can do in situations when we could just as easily do it by hand or a basic calculator.  But the real benefit of R comes into play when working with much larger datasets.

By the end of the section, you'll know 
\begin{enumerate}
\item How to create a vector and provide it a variable name
\item Modify a vector by adding new items to it
\item Perform algebra with vectors
\item Various functions that tell us certain features of vectors, like how long it it, it's largest/smallest value, the sum of all the entries, the mean of the entries
\item How to extract entries with certain properties, e.g. the 23 entree, or the entries larger than 5, etc.  
\end{enumerate}

% ~~~~~~~~~~~~~~~~~~~~~~~~
\subsection{Motivation}

In many statistical scenarios, we'll work with databases. Think of these like an Excel
spreadsheet. We'll get to working with those in R soon enough, for 
now, here's a simple one we're familiar with, a grade sheet
\begin{center}
  \begin{tabular}{cccc}
    \textbf{Student} & \textbf{HW1} & \textbf{HW2} & \textbf{Exam1} \\
    \hline\hline
    Joe              & 35 & 45 & 84 \\
    Jen              & 42 & 48 & 91 \\
    Jes              & 37 & 42 & 85 \\
    Jon              & 32 & 40 & 78 \\
    $\vdots$ & $\vdots$ & $\vdots$ & $\vdots$ \\
  \end{tabular}
\end{center}

A common task is to want to separate out a column, say ``HW1'', and
work with that because maybe you want to
\begin{itemize}
\item add 5 points to every score because a question wasn't fair
\item divide by the total number of possible points to get a percentage score
\item add columns together to get a total HW grade,
\item etc
\end{itemize}

A column in a database like this is simply a vector.
  
% ~~~~~~~~~~~~~~~~~~~~~~~~~~~~~~~~~~~~
\subsection{Defining Vectors}

A vector is a list of either numbers or of words. There is an
R-function to create a \define{vector},

\begin{rcode}
# generic column function
c()

# list with numbers 1,2,3,4
c(1,2,3,4)

# a list of names, note the quotes
c("jim", "james", "jimmie") 
\end{rcode}

Notice those lies that start with a `\#`?  Those are what computer programmers call \define{comments}.  That hash character tells the computer to ignore anything written behind it. This gives the human reading the code a chance to make notes right inside the code in a way that does not cause the computer to make an error.

We just filled a vector with names surrounded by quotations

\begin{rcode}
  c("jim", "james", "jimmie")
\end{rcode}

This ensures that the vector is filled with what are called
\define{strings}, which is a computer science way of saying a sequence
of letters or numbers. If you didn't include the quotations, the
computer reads them as variables. If you never defined a variable
named \texttt{jim}, or \texttt{james}, or \texttt{jimmie}, then you'd
get an error.

\begin{rcode}
  c(jim, james, jimmie) # should produce an error
\end{rcode}

But if you define these variables, then your vector will be filled
with those values.

\begin{rcode}
  c("x", "y", "z") # vector with the letters in each position
  c(x,y,z) # error b/c we never defined variables x,y,z

  # let's define x,y,z as variables 
  x <- 5 
  y <- 12 
  z <- 13 
  c(x,y,z) #no longer an error

    c("x", "y", "z") # still a vector with strings
  \end{rcode}
    
We can name columns to more easily refer to them

\begin{rcode}
  column_1 <- c(5,3,5,7,25,234)
  column_2 <- c("naomi", "wilshi")
  primes <- c(2,3,5,7,11)
  evens <- c(2,4,6,8,10,12,14,16)
\end{rcode}

Then to view them, compile just the name

\begin{rcode}
  column_1 
  column_2 
  primes 
  evens 
\end{rcode}

Suppose you're studying mice, you did an experiment with five mice and
you weighed each. So you created two vectors

\begin{rcode}
  names <- c("mouse_01", "mouse_02", "mouse_03", "mouse_04", "mouse_05")
  weight_g <- c(50,60,65,82,67)
\end{rcode}

If we continued the experiment and want to add another mouse and its
weight to our vector, we can do so by putting our previous vector
names into a new vector and name that.

\begin{rcode}
  names <- c(names, "mouse_06") # note variable in first spot, string in second
  weight_g <- c(weight_g,39)
  
  # print those
  names
  weight_g
\end{rcode} 

Note how these overrode the orginal columns \texttt{names} and \texttt{weight\_g}.  So if you are worried about making a mistake, it's a good practice to give the updated vectors a new name

\begin{rcode}
  # original vectors
  names <- c("mouse_01", "mouse_02", "mouse_03", "mouse_04", "mouse_05")
  weight_g <- c(50,60,65,82,67)
  
  # update them with new mouse info    
  names_v2 <- c(names, "mouse_06") # note variable in first spot, string in second
  weight_g_v2 <- c(weight_g,39)
  
  # now we still have the original
  names
  weight_g
\end{rcode} 

% ~~~~~~~~~~~~~~~~~~~~~~~~~~~~~~~~~~~~
\section{Vector Arithmetic}

In this section, we are going to learn how to add, divide, multiply, and divide. The difference between this and typical algebra is that, typically, you do them to a pair of numbers.  Here, we're going to (1) do the algebra with a number and a vector, and also (2) with two vectors.
  
The the following. Give a variable name to any vector with only
numbers (i.e. no strings) of any length.  Then to that vector add a
number to it, subtract a number from it, multiply it by a number,
divide it, raise it to some power.


Now try this. Give two variable names to two vectors whose elements
are all numbers (i.e. no strings). The vectors should be be the same
length. Then add them together, subtract them, multiply them together,
and raise one to the power of the other.

Now repeat this, except make one vector have length 3 and the other of
length 4. Repeat again except make one vector length 2 and the other
length 4.  Note what happens.

% ~~~~~~~~~~~~~~~~~~~~~~~~~~~~~~~~~
\subsection{Functions on Vectors}

Just like you can plug numbers into R-functions, there are certain
R-functions that we can plug vectors into.
  
Recall that vectors, for us, will typically be filled with data. Perhaps its a column from a database, e.g. a gradebook. Here are some common things we'd want to be able to do with it.
  
\begin{example}
  \begin{rcode}
    hw1_grades <-  c(51,21,35,15,35,15,23,65,15,35,13,54,2
    6,35,12,45,12,35,58,54,43,49,47,45,42,41,43,46,48,54,42,48,47)
    
    length(hw1_grades) # returns how many items are in the list
    max(hw1_grades) # returns the max grade
    min(hw1_grades) # returns the min grade
    sum(hw1_grades) # adds all the grades together
    mean(hw1_grades) # returns the mean grade
    summary(hw1_grades) # a convenient way to get a lot of information about how the grades are distributed
  \end{rcode}
\end{example}
  
% ~~~~~~~~~~~~~~~~~~~~~~~~~~~~~~~
\section{Subsetting Vectors}

Another important construction we'd like to be able to perform on a
vector is to look at certain parts of the vector. For example, maybe
we want to see just the 14th entry, or we want all the entrees above
10, or that are not 0.  This is called \emph{subsetting vectors}
because we are trying to extract a subset of the entrees in our vector
  
Here are various ways to get at some important information contained
within a vector.
  
\begin{rcode}
  # create an arbitrary vector
  x <- c(1,5,4,8,6,4,5,3,1,2,4,9,8,7,4,6,5,3,8,1,2,5,3,6,1,2,5,3)
  
  # to see just the 6th entry, replace 6 with any entry
  x[6]
  
  # to see the 3rd, 6th, 10th entry. Note the column function used to return multiply entrees
  x[c(3,6,10)]
  
  # to see all the entrees strictly greater than 5
  x[x > 5]
  
  # to see all entrees less than or equal to 7
  x[ x <= 7]
  
  # to see all entress that are not equal to 4
  x[ x != 4 ]
  
  # combine multiple properties with & (and) and | (or)
  x[ x>5 & x<=7] # returns 5<x<=7, we do need to use the & here
  x[ x < 3 | x >=8] # returns x str. less than 3 or at least 8    
\end{rcode}

% ~~~~~~~~~~~~~~~~~~~~~~~~~~
\subsection{Exercises}

\begin{q}
  Create a variable (any name you like) from the following vector
  (copy and paste, but make sure it's all on one line):

\begin{rcode}
  c(35,16,64,35,43,25,95,12,62,43,27,81,32,
  46,25,29,93,105,56,68,48,72,83,72,42,29,12,61,
  54,82,53,19,8,68,1,0,29,14,82,86,94,62,52,18,68,75,84)
\end{rcode}

  Using R, what command will print out the values that are
  \begin{enumerate}
  \item greater than 30
  \item not equal to 82
  \item between 45 and 75.
  \end{enumerate} 
\end{q} 

% ~~

\begin{q}
  The number of birds banded at a series of sampling sites has been
  counted by your field crew and entered into the following
  vector. Counts are entered in order and sites are numbered starting
  at one. Cut and paste the vector into your assignment and then
  answer the following questions by printing them to the screen (again
  be certain everything is on the same line). Some
  R functions that will come in handy include \texttt{length(), max(), min(),
  sum()}, and \texttt{mean()}.

\begin{rcode}
number_of_birds <- c(28, 32, 1, 0, 10, 22, 30, 19, 145, 27,36, 25, 9, 38, 21, 12, 122, 87, 36, 3, 0, 5, 55, 62, 98, 32,900, 33, 14, 39, 56, 81, 29, 38, 1, 0, 143, 37, 98, 77, 92, 83, 34, 98, 40, 45, 51, 17, 22, 37, 48, 38, 91, 73, 54, 46, 102, 273, 600, 10, 11)
\end{rcode}

What R command will give the answer the questions?
  \begin{enumerate}
  \item How many sites are there?   
  \item How many birds were counted at site 42?   
  \item What is the total number of birds counted across all of the
    sites?   
  \item What is the smallest number of birds counted?   
  \item What is the largest number of birds counted?    
  \item What is the mean number of birds seen at a site?   
  \end{enumerate} 
\end{q} 

% ~~

\begin{q}
  You have data on the length, width, and height of 10 individuals of
  the yew Taxus baccata (a type of shrub) stored in the following vectors:

\begin{rcode}
length <- c(2.2, 2.1, 2.7, 3.0, 3.1, 2.5, 1.9, 1.1, 3.5, 2.9)
width <- c(1.3, 2.2, 1.5, 4.5, 3.1, 1.7, 1.8, 0.5, 2.0, 2.7)
height <- c(9.6, 7.6, 2.2, 1.5, 4.0, 3.0, 4.5, 2.3, 7.5, 3.2)
\end{rcode}

  What R command will provide the following?
  \begin{enumerate}
  \item The volume of each shrub (i.e., the length times the width
    times the height). storing this in a variable will make some of
    the next problems easier
  \item The total volume of all of the shrubs.
  \item A vector of the height of shrubs with lengths greater than
    2.5.
  \item A vector of the height of shrubs with heights greater than 5.
  \item A vector of the heights of the first 5 shrubs.
  \item A vector of the volumes of the first 3 shrubs.
  \end{enumerate}  
\end{q}


%\section{spreadsheets}
%\label{sec:s}

